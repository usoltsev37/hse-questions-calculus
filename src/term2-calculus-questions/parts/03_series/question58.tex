\Subsection{Билет 58: ! Поточечная и равномерная сходимость рядов. Остаток ряда. Критерий Коши. Необходимое
условие равномерной сходимости ряда.}

\begin{remark}
    Момент с лекции: \href{https://www.youtube.com/watch?v=KfeUfFaKZHo}{youtu.be} \newline
    Записи Александра Игоревича c лекции: \href{https://drive.google.com/file/d/1RjjQ-4fzkRGXTJR-QFnaVCY-DEZRW8Do/view}{drive.google}
\end{remark}

\begin{definition}\slashns
	
	$u_n : E \to \mathbb{R}(\mathbb{C})$
	
	$\sum\limits_{n = 1}^{\infty} u_n(x)$ -- функциональный ряд
	
	$S_n(x) := \sum\limits_{k = 1}^{n} u_k(x)$ -- частичная сумма.
	
	Если $S_n$ поточечно сходится к $S$,то ряд поточечно сходится, если $S_n \rightrightarrows S$, то ряд равномерно сходится.
\end{definition}

\begin{definition}\slashns
	
	Пусть ряд $\sum\limits_{n = 1}^{\infty} u_n(x)$ сходится поточечно
	
	$r_n(x) := \sum\limits_{k = n + 1}^{\infty} u_k(x) = S(x) - S_n(x)$ -- остаток функции ряда.
\end{definition}

\begin{theorem}\slashns
	
	$\sum\limits_{n = 1}^{\infty} u_n(x)$ равномерно сходится на $E$
	
	$\iff r_n \rightrightarrows 0$ на $E$.
\end{theorem}

\begin{proof}\slashns
	
	$\sum\limits_{n = 1}^{\infty} u_n(x)$ -- равномерно сходится $\iff S_n \rightrightarrows S$ на $E \iff r_n=S-S_n \rightrightarrows 0$
\end{proof}

\begin{theorem}[Критерий Коши]\slashns
	
	$\sum u_n(x)$ равномерно сходится на $E$
	
	$\iff \forall \epsilon > 0 \;\; \exists N \;\; \forall n > N \;\; \forall p \in \N \;\; \forall x \in E \;\; |\sum\limits_{k = n+1}^{n+p} u_k(x)| < \epsilon$
\end{theorem}

\begin{proof}\slashns
	
	$\sum u_n(x)$ равномерно сходится $\iff S_n \rightrightarrows S$ на $E$
	
	$\iff \forall \epsilon > 0 \;\; \exists N \;\; \forall m,n > N \;\; \forall x \in E \;\; |S_m - S_n| < \epsilon$
	
	$\forall \epsilon > 0 \;\; \exists N \;\; \forall n > N \;\; \forall p \in \N \;\; \forall x \in E \;\; |S_{n+p} - S_n| < \epsilon$
	
	$|\sum\limits_{k = n+1}^{n+p} u_k(x)| = |\sum\limits_{k = 1}^{n+p} u_k(x) - \sum\limits_{k = 1}^{n} u_k(x)| = |S_{n+p} - S_n|$
\end{proof}

\begin{consequence}[(Необходимое условие сходимости функции ряда)]\slashns

	Если ряд $\sum u_n(x)$ равномерно сходится, то $u_n \rightrightarrows 0$.
\end{consequence}

\begin{proof}\slashns
	
	Возьмем критерий Коши и $p=1$.
	
	$\forall \epsilon > 0 \;\; \exists N \;\; \forall n > N \;\; \forall x \in E \;\; |u_{n+1}(x)| < \epsilon$
	
	Это определение равномерной сходимости $u_n \rightrightarrows 0$.
\end{proof}

\begin{remark}\slashns
	
	\begin{enumerate}
		\item
		Если $\exists x_n \in E$, для которой $u_n(x_n) \not\to 0$, то $\sum u_n(x)$ не сходится равномерно.
		\item Из того, что ряд $\sum u_n(x_n)$ расходится ничего не следует
	\end{enumerate}
\end{remark}

\begin{example}\slashns
	
	$u_n(x) = \begin{cases}
	\frac1n & \text{ при } x \in [\frac{1}{n+1}, \frac1n)\\
	0 & \text{ иначе }
	\end{cases}$
	
	$\sum u_n(\frac{1}{n+1}) = \sum \frac1n$ -- расходится.
\end{example}
