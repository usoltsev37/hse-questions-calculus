\Subsection{Билет 53: Произведение $\prod_{n=1}^\infty \frac{p_n}{p_n - 1}$ и ряд $\sum_{n=1}^\infty \frac{1}{p_n}$}
Тут под $p_n$ поздразумевается $n$-ое простое число.

\begin{statement} \thmslashn

  Произведение $\prod\limits_{n=1}^\infty \frac{p_n}{p_n-1}$ расходится
  \begin{proof} \thmslashn
    
    Для начала проведу некие не совсем формальные рассуждения, далее их формализую. Итак, неформальная часть:
    $$\frac{p_n}{p_n - 1} = \frac{1}{1 - \frac{1}{p_n}} = \sum_{k=0}^\infty\frac{1}{p_n^k}$$
    В начале просто несколько иначе переписал член произведения. Затем заметил, что это сумма бесконечно убывающей геометрической прогрессии. Тогда наше исходное произведение переписывается в такое произведение сумм:
    $$\prod_{n=1}^\infty\sum_{k=0}^\infty\frac{1}{p_n^k}$$
    <<Раскроем>> скобки в этом произведении и получим сумму всеовозможных произведений выражений вида $\frac{1}{p_n^k}$, а именно:
    $$\sum \frac{1}{\prod\limits_k^\infty p_{k=1}^{\alpha_k}}$$
    А с алгебры первого модуля нам известно, что каждое натуральное число единственным образом представляется в виде $\prod_{k=1}^\infty p_k^k$ и при этом единтвенным образом поэтому наша сумма это в точности это:
    $$\sum_{n = 1}^\infty\frac{1}{n}$$
    Получили гармонический ряд, а он, как известно, расходится. Почему же я написал слово <<раскроем>> в кавычках? Все потому, что раскрывать бесконечное произведение бесконечных сумм может быть не совсем законно, как минимум не ясно почему законно, поэтому пришло время формализовать всё то, что я напиал выше:
    $$P_n := \prod_{t=1}^n\sum_{k=0}^\infty\frac{1}{p_t^k} \geqslant \prod_{t=1}^n\sum_{k=0}^n\frac{1}{p_t^k}$$
    Раскроем скобки, получим суммы таких слагаемых $\frac{1}{p_1^{\alpha_1}p_2^{\alpha_2}...}$, где все $\alpha_i \leqslant n$, и все $p_i \leqslant n$, что означает, что там точно будут все дроби вида $\frac{1}{i}$, где $i \leqslant n$ ($i$ - натуральное, если вдруг по каким-то причнам это неочевидно). Тогда для $P_n$ имеем следующее неравнество (уже имеем все такие слагаемые, есть еще какие-то сверху, на них забьем):
    $$P_n \geqslant \sum_{m=1}^n\frac{1}{m}$$
    Заметим, что этот ряд расходится, поэтому и произведение из условия расходится.
  \end{proof}
  \begin{remark} \thmslashn

    $P_n \geqslant \ln n + \mathcal{O}(1)$
    \begin{proof} \thmslashn
      
      
      Мы все прекрасно знаем, что гармонический ряд эквивалентен $\ln n + \gamma + o(1)$. Мы уже показали, что наш ряд больше гармонического ряда, а $\gamma + o(1)$ можно записать в виде $\mathcal{O}(1)$ (потому что постоянная Эйлера и $o(1)$ - какое-то ограниченное выражение). 
    \end{proof}
  \end{remark}
\end{statement}
\begin{theorem} \thmslashn
  
  Ряд $\sum\limits_{n=1}^\infty\frac{1}{p_n}$ расходится
  \begin{proof} \thmslashn

    Из расходимости произведения $\prod\limits_{n=1}^\infty\frac{p_n}{p_n-1}$ ряд из логарифмов $\sum\limits_{n=1}^\infty\ln\frac{p_n}{p_n-1}$ тоже расходится. Посмотрим на один такой логарифм:
    $$\ln\frac{x}{x-1} =  \ln\frac{1}{1-\frac{1}{x}} = -\ln\left(1-\frac{1}{x}\right) \leqslant \frac{2}{x}$$
    Первые два равенства - очевидные, последнее неравенство следует из следующего факта: $\ln(1-t)\geqslant-2t$ при достаточно маленьких $t$ (не верите - дифференцируйте), поэтому выполняется при достаточно больших $x$. Значит, первые члены, для которых неравенство не выполняется, можем оценить какой-то константой $C$ (их конечное число), а для остальных по неравенству:
    $$\sum_{n=1}^\infty \ln \frac{p_n}{p_n -1}\leqslant C + \sum_{n=1}^\infty\frac{2}{p_n}$$
    $$\sum_{n=1}^\infty \ln \frac{p_n}{p_n -1} - C\leqslant \sum_{n=1}^\infty\frac{2}{p_n}$$
    Получилось, что подперли ряд из $\frac{2}{p_i}$ расходящимся рядом, отсюда следует расходимость ряда $\sum\limits_{n=1}^\infty\frac{1}{p_n}$.
  \end{proof}
  \begin{remark} \thmslashn

    На самом деле
    $$\ln\frac{x}{x-1} \sim \frac{1}{x}$$
    (потому что $-\ln\left(1 - \frac{1}{x}\right) \sim \frac{1}{x}$), поэтому
    $$\sum\limits_{n=1}^\infty\frac{1}{p_k} \sim \sum\limits_{n=1}^\infty \ln\frac{p_k}{p_k - 1}$$
    (теорема Штольца утверждает (гугл в помощь, если что), что если каждое слагаемое (то есть $a_n - a_{n-1}$) эквивалентно, то и суммы ($a_n$) эквивалентны), то есть
    $$\sum\limits_{n=1}^\infty\frac{1}{p_k} \sim \ln P_n \geqslant \ln(\ln n + \mathcal{O}(1)) \geqslant \ln\ln n + \mathcal{O}(1)$$
    \begin{statement} \thmslashn

      $$\sum_{n=1}^\infty\frac{1}{p_n} \sim \ln\ln n$$
      \begin{proof} \thmslashn

        Без доказательства.
      \end{proof}
    \end{statement}
  \end{remark}
\end{theorem}
