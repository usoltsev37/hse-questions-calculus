\Subsection{Билет 96: Наибольшее и наименьшее значение квадратичной формы на единичной сфере. Формула для нормы матриц.}

\begin{theorem} \thmslashn

Наибольшее (наименьшее) значение на квадратичной формы на сфере равно наибольшему (наименьшему) собственному числу матрицы, задающей данную форму.
  \begin{proof} \thmslashn
  
    Будем находить наименьшее и наибольшее значение с помощью метода множителей Лагранжа. Для этого надо соорудить фнукцию, которую мы будем минимизировать и функцию-уравнение, которое будет является условием, при котором мы ищем минимум. $f(x) = Q(x) = \sum\limits_{i, j = 1}^{n}{c_{ij}x_ix_j}$ - функция, которую хотим минимизировать (максимизировать).
    $\Phi(x) = ||x||^2 - 1 = 0 = \sum\limits_{i = 1}^{n}{x_i^2} - 1$ - функция-условие. Данному условию удовлетворяют точки, норма которых равна 1 $\Leftrightarrow$ лежат на единичной сфере.
    \par Стоит заметить, что метода Лагранжа позволяет найти точки-экстрмумы, а наша текущая задача состоит в том, чтобы найти наибольшее и наименьшее значение функции, но так как множество на которому мы рассматриваем нашу функцию - компакт (единичная сфера - компакт), то по теореме Вейерштрасса, функция достигает своего наибольшего и наименьшего значения на данном множетсво, значит достаточно рассмотреть точки-экстремумы.
    \par Составим функцию $F(x) = f(x) - \lambda\Phi(x)$. Такая функция называется \textit{функцией Лагранжа}. Она удобна, тем, что по теореме Лагранжа (об этом методе), для точек условного экстремума $\grad f$ линейно зависим с $\grad\Phi_1, \dots, \grad\Phi_n$. Значит $\exists \lambda_1, \dots, \lambda_n : \grad f = \sum\limits_{i = 1}^{n}\lambda_i\grad\Phi_i$. Чтобы найти лямбды можно приравнять нулю все частные производные $F$, так как это и означает линейную зависимость $\grad f, \grad \Phi_1, \dots, \grad \Phi_n$.
    \par Решим уравения в нашей задаче, для этого продифференцируем $F$. 
    \[ F_{x_k}' = \left(\sum\limits_{i, j = 1}^{n}{a_{ij}x_ix_j} - \lambda\sum\limits_{i = 1}^{n}{x_i^2}\right)_{x_k}' = a_{kk}\cdot 2x_k + \sum\limits_{i\neq k}{a_{ik}x_{i}} + \sum\limits_{i\neq k}{a_{ki}x_{i}} - 2\lambda x_k = 0 \]
    Производная $f$ по $x_k$ имеет такой вид, так как $f$ содержит слагемые вида: $a_{kk}x_k^2, a_{ki}{x_kx_i}, a_{ik}x_ix_k$. \par
    Сгруппируем слагаемые:
    \[ F_{x_k}' = 2\sum\limits_{i = 1}^{n}a_{ik}x_i - 2\lambda x_k = 0 \]
    Заметим, что $\sum\limits_{i = 1}^{n}a_{ik}x_i = Ax$, где $A$ - матрица квадратичной формы. \par
    Тогда уравнение приобретает вид:
    \[ Ax - \lambda x = 0 \]
    Это ни что иное, как равенство для собственных векторов матрицы $A$. Значит точки эксремума это собственные вектора, а $\lambda$-ы равны собственным числам. Посчитаем значение $f(x) = Q(x)$ в данных точках. 
    \[ Q(x) = \left<Ax, x\right> = \left<\lambda x, x\right> = \lambda \left<x, x\right> = \lambda ||x||^2 = \lambda \]
    Значит наибольшее (наименьшее) значение $Q(x)$ равно набиольшему (наименьшему) собственному числу и достигается оно на соответсвующем собственном векторе.
  \end{proof}

  \begin{remark} \thmslashn

    Можно было не требовать, чтобы радиус сферы был равен 1, если положить радиус сферы равным $R$, то изменились бы только последнии выкладки в доказательстве, а именно экстремальное значение формы стало бы равным $R^2 \cdot \lambda$ (так как тогды бы $||x|| = R$).
  \end{remark}
  

\end{theorem}

\begin{consequence} \thmslashn

  $A : \R^n \mapsto \R^n$, тогда $||A|| = \max \{\sqrt{\lambda} : \lambda - \text{ собственное число } A^{\top}A \}$

  \begin{proof} \thmslashn

    Оценим норму в квадрате:    
    \[ ||A||^2 = \max\limits_{||x|| = 1}{||Ax||^2} = \max\limits_{||x|| = 1}\left<Ax, Ax\right> = \max\limits_{||x||=1}\left<A^{\top}Ax, x\right> \]
    \begin{enumerate}
      \item Воспользовались одним из определений нормы оператора 
      \item Воспользовались $||a||^2 = \left<a, a\right>$ 
      \item Воспользовались фактом $\left<a, Ab\right> = \left<A^{\top}a, b\right>$
    \end{enumerate}
    $A^{\top}A$ - симмитричная матрица $\implies$ задает какую-то квадратичную форму. Не сложно заметить, что $\left<A^{\top}Ax, x\right> = Q(x)$, для формы $Q$, которая задается матрицей $A^{\top}A$.
    Получили выражение из предыдущей теормемы 
    \[ \max\limits_{||x|| = 1}\left<A^{\top}Ax, x\right> = \max \text{ собственное число} \]
    Так как оценивали норму в квадрате, то 
    \[ ||A|| = \max \sqrt{\lambda} \]
  \end{proof}
\end{consequence}
