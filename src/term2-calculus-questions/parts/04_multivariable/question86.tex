\Subsection{Билет 86: Многомерная формула Тейлора с остатком в форме Пеано. Полиномиальная формула.}

\begin{theorem}
	
	Формула Тейлора с остатком в форме Пеано.
	
	$D \subset R^n$, $D$ - открытое множество. $f \in C^{r}(D)$ $a \in D$. $h := x - a$
	
	Тогда при $x \rightarrow a$ $f(x) = \sum\limits_{|k| \leq r}\dfrac{f^{(k)}(a)}{k!}h^k + o(||h||^r)$.
	
	\begin{remark}
		
		Данная формула является следствием из теоремы о многомерной формуле Тейлора с остатком в форме Лагранжа (см. билет 85).
	\end{remark}
	
\end{theorem}

\begin{proof}
	
	Запишем формулу из теоремы о формуле Тейлора с остатком в форме Лагранжа для $r - 1$.
	
	$f(x) = \sum\limits_{|k| \leq r - 1} \dfrac{f^{(k)}(a)}{k!}h^k + \sum\limits_{|k| = r} \dfrac{f^{(k)}(a + \theta h)}{k!}h^k = \sum\limits_{|k| \leq r}\dfrac{f^{(k)}(a)}{k!}(h)^k + \sum\limits_{|k| = r}\dfrac{f^{(k)}(a + \theta h) - f^{(k)}(a)}{k!}h^k$.
	
	Осталось понять, что второе слагаемое и есть $ o(||h||^r)$.
	
	\textbf{Наблюдение 1.}
	$\dfrac{|h^k|}{||h||^r} \leq 1$.
	
	Это верно, так как в числителе мы взяли какие-то координаты вектора $h$ в количестве $r$ штук и умножили друг на друга. В знаменателе мы взяли длину вектора в том же количестве и перемножили. Каждая координата меньше длины. $|h_i| \leq ||h||$. 
	
	\textbf{Наблюдение 2.}
	Осталось доказать, что коэффициенты стремятся к нулю:
	
	$f^{(k)}(a + \theta h) - f^{(k)}(a) \rightarrow 0$ при $h \rightarrow 0$. Это следует из непрерывности производной соответствующего порядка. А сама непрерывность выполняется по условию теоремы.
	
	\begin{remark}
		
		"А на самом деле, если повозиться посильнее, то можно выкинуть требование непрерывности последней производной".
		
		Это значит, что достаточно $r-$той дифференцируемости в точке $a$.
		
		"Но мы не будем лезть в эти подробности (спасибо!)"
	\end{remark}
	
\end{proof}

\begin{consequence} (Полиномиальная формула)
	
	Формула для возведения суммы в $r$-тую степень. 
	
	$(x_1 + x_2 + ... + x_n)^r = \sum\limits_{|k| = r} {r\choose {k_1, k_2, ..., k_n}} x_1^{k_1}...x_n^{k_n}$.
\end{consequence}

\begin{proof}
	
	$f(x_1, x_2, ..., x_n) = (x_1, x_2, ...,x_n)^r =: (g(x))^r$
	
	Подставим это в формулу Тейлора. Для этого поймем, как выглядит производная.
	
	$\dfrac{\delta f}{\delta x_i} = rg^{r - 1}(x)\dfrac{\delta g}{\delta x_i} = rg^{r - 1}(x)$ (так как производная $g$ по $x$ - единица). Получается, частная производная не зависит от координаты, по которой считаем и считается как обычная производная от функции $g$. \\ Значит, производная $r$-того порядка = $\dfrac{\delta^r f}{\delta x_{i_1} ... \delta x_{i_r}} = r!$. А производная, например, $r + 1$-го порядка = 0.
	
	Запишем формулу Тейлора с остатком в виде в форме Лагранжа для $r$. Сразу заметим, что остатка не будет, т.к. $r-$тая производная = 0.
	
	$f(x) = \sum\limits_{|k| \leq r} \dfrac{f^{(k)}(0)}{k!} x^k$.
	
	Заметим, что производная порядка $< r$ в нуле будет = 0, т.к. в формуле у нас останется $g$ в какой-то ненулевой степени, а $g$ в нуле = 0 $\Rightarrow$.
	
	$f(x) = \sum\limits_{|k| \leq r} \dfrac{f^{(k)}(0)}{k!} x^k = \sum\limits_{|k| = r}\dfrac{r!}{k!}x^k$. А это и есть то, что было обещано в начале. Доказали.
	
\end{proof}