\Subsection{Билет 79: Теорема Лагранжа для векторнозначных функций.}

\begin{theorem} \thmslashn

    $f : [a, b] \mapsto \R^m$ непрерывна и дифференцируема на $(a, b)$. Тогда $\exists{c}\quad \in (a, b)$, такая что $\left\| f(b) - f(a) \right\| \le \left\| f'(c)\right\|(b - a)$
    \begin{proof} \thmslashn

	$\phi(x) := \left< f(x), f(b) - f(a)\right> : [a, b] \mapsto \R$

	$\phi(x)$ удовлетворяет условию одномерной теоремы Лагранжа

	$\exists{c} \quad \in (a, b)$, т.ч. $\phi(b) - \phi(a) = \phi(c')(b - a) = \left< f'(c), f(b) - f(a)\right>(b - a)$

	$\phi'(x) = \left< f'(x), f(b) - f(a)\right> + \left< f(x), (f(b) - f(a))'\right> = \left< f'(x), f(b) - f(a)\right>$

        $\phi(b) - \phi(a) = \left< f(b), f(b) - f(a)\right> - \left< f(a), f(b) - f(a)\right> = \left< f(b) - f(a), f(b) - f(a)\right> = \left\| f(b) - f(a) \right\|^2$

        $\left\| f(b) - f(a) \right\|^2 = \left< f'(c), f(b) - f(a)\right>(b - a) \le \left\| f'(c) \right\| \left\| f(b) - f(a) \right\| (b - a)$ (Коши-Буняковский)

        $\left\| f(b) - f(a)\right\| \le \left\| f'(c) \right\|(b - a)$
    \end{proof}
    \item Замечание. Равенство может никогда не достигаться 

	$f(x) = (\cos{x}, \sin{x}) : [0, 2\Pi] \mapsto \R^2$
 
	$f(0) = (1, 0) = f(2\Pi)$

	$f(2\Pi) - f(0) = (0, 0) \implies \left\| f(2\Pi) - f(0) \right\| = 0$

	$f'(x) = ((\cos{x})', (\sin{x})') = (-\sin{x}, \cos{x})$

	$\left\| f'(x) \right\| = 1 \implies \left\| f'(c) \right\| (2\Pi - 0) = 2\Pi > \left\| f(2\Pi) - f(0) \right\| = 0$

\end{theorem}

