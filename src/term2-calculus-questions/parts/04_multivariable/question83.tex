\Subsection{Билет 83: Теорема о равенстве частных производных для непрерывно дифференцируемых функций.}



\begin{definition} \thmslashn
	
$f : D \mapsto \mathbb{R}$\quad $D \subset \mathbb{R}^n$\quad $D$ - открыто\\
$f$ - $r$ раз непрерывно дифференцируема ($r$-гладкая), если все её частные производные до $r$-го порядка включительно существуют и непрерывны.\\
Обозначение - $f \in C^r(D)$
\end{definition}

\begin{theorem} \thmslashn
	
$f : D \mapsto \mathbb{R}$\quad $D \subset \mathbb{R}^n$\quad $D$ - открыто $f \in C^r(D)$\\
$i_1, i_2, ..., i_r$ - перестановка $j_1, j_2, ..., j_r$ (этих элементов)\\
Тогда $\frac{\partial^rf}{\partial x_{i_1}\partial x_{i_2}...\partial x_{i_r}} = \frac{\partial^rf}{\partial x_{j_1}\partial x_{j_2}...\partial x_{j_r}}$\\
производные берутся справа налево, т.е. $f_{xy}'' = \frac{\partial^2}{\partial y \partial x}f = \frac{\partial}{\partial y}(\frac{\partial}{\partial x}f)$

\begin{proof} \thmslashn

Перестановка раскладывается на транспозиции, значит, достаточно доказать что\\ $\frac{\partial^rf}{\partial x_{i_1}\partial x_{i_2}...\partial x_{i_k}\partial x_{i_{k+1}}...\partial x_{i_r}} = \frac{\partial^rf}{\partial x_{i_1}\partial x_{i_2}...\partial x_{i_{k+1}}\partial x_{i_{k}}...\partial x_{i_r}}$ (поменяли местами два элемента)\\
Мы знаем, что 
$\frac{\partial^{2}g}{\partial x_{i_k}\partial x_{i_{k+1}}} = \frac{\partial^{2}g}{\partial x_{i_{k+1}}\partial x_{i_{k}}}$ (из билета 82),  где $g = \frac{\partial^{r-k-1}f}{\partial x_{i_{k+2}}...\partial x_{i_r}}$ \\
Тогда 
$\frac{\partial^{k+1}g}{\partial x_{i_1}\partial x_{i_2}...\partial x_{i_k}\partial x_{i_{k+1}}} = \frac{\partial^{k+1}g}{\partial x_{i_1}\partial x_{i_2}...\partial x_{i_{k+1}}\partial x_{i_{k}}}$\\
Значит, $\frac{\partial^rf}{\partial x_{i_1}\partial x_{i_2}...\partial x_{i_k}\partial x_{i_{k+1}}...\partial x_{i_r}} = \frac{\partial^rf}{\partial x_{i_1}\partial x_{i_2}...\partial x_{i_{k+1}}\partial x_{i_{k}}...\partial x_{i_r}}$
\end{proof}
\end{theorem}


\begin{remark}
	
Необходимость условия $f \in C^r(D)$ (производные непрерывны)\\

$f = 
\begin{aligned}
\left\{
\begin{array}{ll}
xy\frac{x^2 - y^2}{x^2 + y^2} &, (x, y) \neq (0, 0) \\
0 &, (x, y) = (0, 0) 
\end{array}
\right.
\end{aligned}
$\\

$f_{x}' = 
\begin{aligned}
\left\{
\begin{array}{ll}
y\frac{x^2 - y^2}{x^2 + y^2} + xy\frac{2x(x^2 + y^2) - 2x (x^2 - y^2)}{(x^2 + y^2)^2} = y\frac{x^2 - y^2}{x^2 + y^2} + xy \frac{4xy^2}{(x^2 + y^2)^2} &, (x, y) \neq (0, 0) \\
\lim\limits_{h\to0}\frac{f(h, 0) - f(0, 0)}{h} = \lim\limits_{h\to0}\frac{0 - 0}{h} = 0 &, (x, y) = (0, 0) 
\end{array}
\right.
\end{aligned}
$\\

$
f_{xy}''(0, 0) = \lim\limits_{k\to0}\frac{f_{x}'(0, k) - f_{x}'(0, 0)}{k} = \lim\limits_{k\to0}\frac{-k - 0}{k} = -1
$

$f_{y}' = 
\begin{aligned}
\left\{
\begin{array}{ll}
x\frac{x^2 - y^2}{x^2 + y^2} + xy\frac{-2y(x^2 + y^2) - 2y (x^2 - y^2)}{(x^2 + y^2)^2} = x\frac{x^2 - y^2}{x^2 + y^2} + xy \frac{-4x^2y}{(x^2 + y^2)^2} &, (x, y) \neq (0, 0) \\
\lim\limits_{h\to0}\frac{f(0, h) - f(0, 0)}{h} = \lim\limits_{h\to0}\frac{0 - 0}{h} = 0 &, (x, y) = (0, 0) 
\end{array}
\right.
\end{aligned}
$\\

$
f_{yx}''(0, 0) = \lim\limits_{k\to0}\frac{f_{y}'(k, 0) - f_{y}'(0, 0)}{k} = \lim\limits_{k\to0}\frac{k - 0}{k} = 1
$\\

$\implies f_{xy}''(0, 0) \neq f_{yx}''(0, 0)$ !!!!

Всё потому что $f_{xy}''$ не является непрерывной в точке $(0, 0)$:\\
$f_{xy}'' = f_{y}'(f_{x}') = 
(y\frac{x^2 - y^2}{x^2 + y^2} + xy \frac{4xy^2}{(x^2 + y^2)^2})_{y}' = 
\frac{x^2 - y^2}{x^2 + y^2} + y\frac{-4x^2y}{(x^2 + y^2)^2} + x\frac{4xy^2}{(x^2 + y^2)^2} + xy\frac{8xy(x^2+y^2)^2 - 8xy^2(x^2+y^2)2y}{(x^2 + y^2)^4} =\\ 
\frac{x^2 - y^2}{x^2 + y^2} + \frac{8x^2y^2\cdot ((x^2+y^2) - 2y^2)}{(x^2 + y^2)^3} = \frac{x^2 - y^2}{x^2 + y^2} + \frac{8x^2y^2}{(x^2 + y^2)^2} + \frac{-16x^2y^4}{(x^2 + y^2)^3}$, $(x, y) \neq (0, 0)$\\
$(x, y) \to 0 \iff (r\cdot \cos{\varphi}, r\cdot \sin{\varphi}) \to 0 \iff r \to 0$\\
$f_{xy}'' = \frac{r^2(\cos^2{\varphi} - \sin^2{\varphi})}{r^2(\cos^2{\varphi} + \sin^2{\varphi})} + \frac{8\cdot r^2\cos^2{\varphi}\cdot r^2\sin^2{\varphi}}{r^4(\cos^2{\varphi} + \sin^2{\varphi})^2} + \frac{-16\cdot r^2\cos^2{\varphi}\cdot r^4\sin^4{\varphi}}{r^6(\cos^2{\varphi} + \sin^2{\varphi})^3} = \\
(\cos^2{\varphi} - \sin^2{\varphi}) + (8cos^2{\varphi}\sin^2{\varphi}) + (-16cos^2{\varphi}\sin^4{\varphi}) \not\to 0$ при $r \to 0$

\end{remark}
