\Subsection{Билет 38: Длина кривой, заданной параметрически(с леммой). Длина графика функции и длина кривой, заданной в полярных координатах.}

В этом билете все пути - в $\R^{m}$ с Евклидовой метрикой.

\begin{definition}[повтор] \thmslashn 

    Пусть $\gamma : [a, b] \mapsto \R^{m}$ - путь.

    Пусть $\gamma_{i} $ - $i$-я координатная функция $\gamma$.

    $\gamma$ называется $r$-гладким путём, если $\forall{i}\quad \gamma_{i}\in C^{r}[a, b]$.

    Просто гладкий - $r = 1$.
\end{definition}

\begin{lemma} \thmslashn

    Пусть $\gamma : [a, b] \mapsto \R^{m}$ - гладкий путь.

    Пусть $\Delta \subset [a, b]$ - отрезок, $\ell(\Delta)$ - его длина.

    $m^{(i)}_{\Delta} := \min\limits_{t\in \Delta} \abs{\gamma_{i}'(t)}$
    
    $M^{(i)}_{\Delta} := \max\limits_{t\in \Delta} \abs{\gamma_{i}'(t)}$

    $m_{\Delta} := \sqrt{\sum\limits_{i=1}^{m} \left(m^{(i)}_{\Delta}\right)^2} $
    
    $M_{\Delta} := \sqrt{\sum\limits_{i=1}^{m} \left(M^{(i)}_{\Delta}\right)^2}$

    Тогда, 
    \[ m_{\Delta} \cdot \ell(\Delta) \le \ell(\left. \gamma\right|_{\Delta}) \le M_{\Delta} \cdot \ell(\Delta) .\]
        \begin{proof} \thmslashn
        
            Пусть $t_{i}$ - разбиение $\Delta$.

            $a_{k}^{(i)} = \gamma_{i}(t_{k}) - \gamma_{i}(t_{k-1})$

            $a_{k} = \rho(\gamma(t_{k-1}), \gamma(t_{k})) = \sqrt{\sum\limits_{i=1}^{m} \left( a_{k}^{(i)} \right)^2 } $.

            По теореме Лагранжа: $\exists{c_{ik}\in (t_{k-1}, t_{k})}\quad a_{k}^{(i)} = \gamma'(c_{ik})(t_{k} - t_{k-1})$.

            Тогда, $\abs{a_{k}^{(i)}} = \abs{\gamma'(c_{ik})}(t_{k}-t_{k-1}) \le M_{\Delta}^{(i)}(t_{k}-t_{k-1})$.

            \begin{equation*}
                \begin{split}
                    a_{k} = \sqrt{\sum\limits_{i=1}^{m} \left( \abs{a_{k}^{(i)}} \right)^2} 
                    &\le \sqrt{\sum\limits_{i=1}^{m} \left(M_{\Delta}^{(i)}\right)^2(t_{k}-t_{k-1})^2}\\
                    &= \sqrt{(t_{k} - t_{k-1})^2 \sum\limits_{i=1}^{m} \left(M_{\Delta}^{(i)}\right)^2 }\\
                    &= M_{\Delta}(t_{k}-t_{k-1})
                \end{split}
            \end{equation*}
            
            Аналогично для $a_{k} \ge m_{\Delta}(t_{k}-t_{k-1})$.

            \[ m_{\Delta} \ell(\Delta) = m_{\Delta} \sum\limits_{k=1}^{n} (t_{k}-t_{k-1}) \le \sum\limits_{k=1}^{n} a_{k} \le M_{\Delta} \sum\limits_{k=1}^{n} (t_{k}-t_{k-1}) = M_{\Delta} \ell(\Delta) .\]

            Перейдём к супремому для $t$:

            \[ m_{\Delta} \ell(\Delta) \le \ell(\left.\gamma\right|_{\Delta}) \le M_{\Delta} \ell(\Delta) . \qedhere\] 
        \end{proof}
\end{lemma}

\begin{theorem}[Длина кривой заданной параметрически] \thmslashn

    Пусть $\gamma : [a, b] \mapsto \R^{m}$ - гладкий путь.

    Тогда
    \[ \ell(\gamma) = \int\limits_{a}^{b} \norm{\gamma'(t)}dt = \int\limits_{a}^{b} \sqrt{\gamma_1'(t)^2 + \gamma_2'(t)^2 + \ldots + \gamma_{n}(t)^2}dt    .\]
    \begin{proof} \thmslashn
    
        Возьмём $t$ - разбиение $[a, b]$.
        
        $d_{k} := t_{k} - t_{k-1}$.

        $m_{k} := m_{[t_{k-1}, t_{k}]}$

        $M_{k} := M_{[t_{k-1}, t_{k}]}$

        Тогда, по лемме значем
        \[ m_{k}d_{k} \le \ell\left( \left. \gamma\right|_{[t_{k-1}, t_{k}]} \right) \le M_{k}d_{k}  .\]

        По линейности интеграла:
        \[ m_{k}d_{k} \le \int\limits_{t_{k-1}}^{t_{k}} \norm{\gamma'(t)}dt  \le M_{k}d_{k}  .\]

        ($m_{k}$ это норма минимума, которая точно не больше нормы произвольного значения, аналогично для $M_{k}$).

        Просумируем по $k$:
        \[ \sum\limits_{k=1}^{n} m_{k}d_{k} \le \ell\left(\gamma\right) \le \sum\limits_{k=1}^{n} M_{k}d_{k}  .\]
        
        \[ \sum\limits_{k=1}^{n} m_{k}d_{k} \le \int\limits_{a}^{b} \norm{\gamma'(t)}dt  \le \sum\limits_{k=1}^{n} M_{k}d_{k}  .\]

        Покажем что $\sum\limits_{k=1}^{n} (M_{k} - m_{k})d_{k} \to 0$ при $|\tau| := \max\limits_{k} d_{k} \to 0$.

        \textbf{Дальше идёт страшная выкладка из конспекта Ани, есть есть док-во красивее, просьба сказать\ldots}


        \begin{equation*}
            \begin{split}
                \sum\limits_{k=1}^{n} (M_{k}-m_{k})d_{k}
                &= \sum\limits_{k=1}^{n} \frac{(M_{k}-m_{k})(M_{k}+m_{k})}{M_{k}+m_{k}}d_{k}\\
                &= \sum\limits_{k=1}^{n} \frac{M_{k}^2-m_{k}^2}{M_{k}+m_{k}}d_{k}\\
                &= \sum\limits_{k=1}^{n} \frac{d_{k}}{M_{k}+m_{k}} \sum\limits_{i=1}^{m} \left(\left(M_{k}^{(i)}\right)^2 - \left(m_{k}^{(i)}\right)^2\right)\\
                &= \sum\limits_{k=1}^{n} \sum\limits_{i=1}^{m} \frac{\left(M_{k}^{(i)}\right)^2 - \left(m_{k}^{(i)}\right)^2}{M_{k}+m_{k}}d_{k}\\
                &= \sum\limits_{k=1}^{n} \sum\limits_{i=1}^{m} \frac{M_{k}^{(i)} + m_{k}^{(i)}}{M_{k}+m_{k}} \left( M_{k}^{(i)} - m_{k}^{(i)} \right)d_{k}\\
                &\le \sum\limits_{k=1}^{n} \sum\limits_{i=1}^{m} \left( M_{k}^{(i)} - m_{k}^{(i)} \right)d_{k} \text{ т. к. евклидова норма всегда больше координаты}\\
                &\le \sum\limits_{k=1}^{n} \sum\limits_{i=1}^{m} \omega_{\gamma'_{i}}(|\tau|)d_{k} \TODO{\text{ какой-то нетрививальный факт }}\\
                &= (b-a)\sum\limits_{i=1}^{m} \omega_{\gamma_{i}'}(|\tau|)
            \end{split}
        \end{equation*}

        $\gamma$ гладкий $\implies$ $\forall{i}\quad \gamma_{i}\in C^{1}[a, b] \implies \gamma_{i}'\in C[a, b] \implies \lim\limits_{|\tau| \to 0} \omega_{\gamma_{i}'}(|\tau|) = 0$.

        Значит, так-как $\abs{a-b} \le \max \{\max a - \min b, \max b - \min a\} $:
        \[ \abs{\ell(\gamma) - \int\limits_{a}^{b} \norm{\gamma'(t)}dt } \le \sum\limits_{k=1}^{n} (M_{k}-m_{k})d_{k} \to 0 \text{} .\]

        Заметим, что значения длины и интеграла не зависят от выбранного разбиения.

        Если предположить что они не равны, то можем по их разности выбрать такое $t$, что сумма получится меньше. Противоречие, значит равны.
    \end{proof}
\end{theorem}
\begin{consequence} \thmslashn

    \[ \ell(\gamma) \le (b-a) \cdot  \max\limits_{t} \norm{\gamma'} .\] 
\end{consequence}
\begin{definition} \thmslashn 

    Пусть $f : [a, b] \mapsto \R$. Длиной графика $f$ называется длина пути 
    \[ \gamma(t) = \begin{bmatrix} t\\ f(t) \end{bmatrix}  .\] 
\end{definition}
\begin{consequence} \thmslashn

    Если $f\in C[a, b]$, то длина графика $f$ равна
    \[ \int\limits_{a}^{b} \sqrt{1 + f'(t)^2}dt    .\] 
    \begin{proof} \thmslashn
    
        \[ \gamma_1(t) = t \implies \gamma_1'(t) = 1 .\]
        \[ \gamma_2(t) = f(t) \implies \gamma_2'(t) = f'(t).\qedhere\] 
    \end{proof}
\end{consequence}
\begin{definition} \thmslashn 

    Если функция задана в полярных координатах как $r : [\alpha, \beta] \mapsto \R$, то задаваемый ей путь - 
    \[ \gamma(\phi) = \begin{bmatrix} r(\phi)\cos \phi\\ r(\phi)\sin \phi \end{bmatrix}  .\] 
\end{definition}
\begin{consequence} \thmslashn

    Пусть функция в полярных координатах задана как  $r\in C[\alpha, \beta]$. Тогда длина её графика:
    \[ \int\limits_{\alpha}^{\beta} \sqrt{r(\phi)^2 + r'(\phi)^2}   .\]
    \begin{proof} \thmslashn
    
        \[ \gamma_1(\phi) = r(\phi)\cos \phi \implies \gamma_1'(\phi) = -r(\phi)\sin \phi + r'(\phi)\cos \phi .\] 
        \[ \gamma_2(\phi) = r(\phi)\sin \phi \implies \gamma_2'(\phi) = r'(\phi)\sin \phi + r(\phi)\cos \phi .\]
        \[ \gamma_1'(\phi)^2 + \gamma_2'(\phi)^2 = (r(\phi)^2 + r'(\phi)^2)(\cos^2 \phi +\sin^2 \phi) = r(\phi)^2 + r'(\phi)^2.\qedhere\] 
    \end{proof}
\end{consequence}
\begin{definition} \thmslashn 

    Кривая называется спрямляемой, если её длина конечна.
\end{definition}
\begin{consequence} \thmslashn

    Для любой кривой, $f(t) := \ell\left( \left. \gamma\right|_{[a, t]} \right) $ - монотонная функция.

        $f(t)$ непрерывна тогда и только тогда, когда кривая спрямляема. Без доказательства.
\end{consequence}
